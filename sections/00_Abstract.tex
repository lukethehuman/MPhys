% Background.
In a close binary system, mass can be stripped from the surface of one star and accreted onto its companion.
This can alter the stellar wind chemical yields of both stars due to the resulting change in surface composition and direct ejection of material during the mass transfer.
% Al-26
Aluminium-26 ($^{26}$Al), an isotope produced in core hydrogen-burning but destroyed in core helium-burning, is not usually ejected in large quantities by massive single stars before the supernova.
In binary systems, however, mass transfer before the end of helium-burning can eject deep layers of $^{26}$Al.
% What we did.
In this study, we modelled the evolution of non-rotating massive binaries to investigate $^{26}$Al yields while varying period P and mass transfer efficiency $\beta$.
% What we found.
The initial period of the system was found to affect how soon, if at all, mass transfer occurred.
$^{26}$Al yields for binary systems with initial masses 20 \& 18 M$_{\sun}$ were found to increase due to mass transfer between hydrogen and helium-burning in the mass-losing star. Yields were not significantly affected if mass transfer occurred during helium-burning.
In addition, yields were not affected in systems with initial masses 50 \& 45 M$_{\sun}$ since stars of this mass eject their envelope as they evolve into Wolf-Rayet stars.
Varying $\beta$ decreased yields for the mass-losing star as a higher fraction was accreted onto the companion.
With more accreted material, companion stars were found to have increased yields which, if allowed to complete their evolution, outweighed the reduction in yields for the mass-losing star.