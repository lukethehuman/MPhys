\section{Conclusion} % Approx word count = 250 words (total = 3500)

% Basic Result
The effects of mass transfer via RLOF on $^{26}$Al wind yields from non-rotating massive binary stars was investigated.
The initial orbital period was found to determine when RLOF occurred in the star's evolution (i.e. whether mass transfer was case B, C, etc.) but had minimal effect on yields beyond that.
For case B mass transfer, yields were found to increase by a factor of $\sim$4 for a mass-losing 20 M$_{\sun}$ star at mass transfer efficiency $\beta$ = 0.5.
For case C mass transfer, the yields were comparable to those of their single star counterparts due to a combination of reduced mass loss and increased $^{26}$Al decay.
If the star would normally evolve into a Wolf-Rayet star (initial mass > $\sim$20 M$_{\sun}$), $^{26}$Al yields were instead decreased due to accretion of material that would otherwise be ejected.
Yields were also found to increase for the mass-gaining star due to increased surface $^{26}$Al abundance and increased mass loss rates resulting from the transfer of angular momentum.
Mass-gaining stars that would normally evolve into WR stars were found to have their main sequence lifespan extended, but as above their yields remained largely unaffected.

By increasing $\beta$, it was found that the reduced yields from mass-losing stars due to increased accretion onto a companion was ultimately outweighed by the increase in the companion's yields (under the assumption that the companion completes its evolution without continued binary interaction).
The total yield for a 20 \& 18 M$_{\sun}$ system was found to increase by a factor $\sim$3 from 1.54$\times$10$^{-4}$ ($\beta$ = 0) to 4.75$\times$10$^{-4}$ M$_{\sun}$ ($\beta$ = 1).
This increase was found to be non-linear, and was especially significant for companion stars that gained sufficient mass to become WR stars.